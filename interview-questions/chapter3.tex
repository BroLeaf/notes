\chapter{Data structures}

\section{Stack \& queue}
\begin{Exercise}[title=Implement a stack using queues]
Implement a stack by using queues.
\end{Exercise}
\begin{Answer}
Use two queues. 
The first queue stores all top elements in the stack with size at most $O(\sqrt{n})$. 
The second queue stores all other elements. 
Both queues maintain stack order. 

For the pop operation, if the first queue is not empty, dequeue from the first queue.
Otherwise, dequeue from the second queue.

For push operation, enqueue to the first queue. 
Then, use queue operations to reorder the elements in the first queue to maintain stack order. 
If the size of the first queue is larger than $\sqrt{n}$, then insert all elements in the second queue to the first queue and swap first queue with the second queue.

\subparagraph{Complexity analysis} The pop operation takes $O(1)$ time in the worst case. The push operation takes $O(\sqrt{n})$ amortized time.
In general, a stack can be simulated by $k$ queues so that each of push and pop operations takes $O(n^{\frac{1}{k}})$ amortized time.
\begin{remark}
The above solution is from H{\"{u}}hne's paper~\cite{Huhne1993}.\footnote{\url{http://cstheory.stackexchange.com/questions/2562/one-stack-two-queues}.} This problem appears on \href{https://leetcode.com/problems/implement-stack-using-queues/}{LeetCode}.
\end{remark}
\end{Answer}

\subsection{Other problems}
\begin{enumerate}
\item implement a queue using stacks: this problem has been used in Ph.D. qualify exam in Princeton.\footnote{\url{https://rjlipton.wordpress.com/2010/04/25/natural-natural-and-natural/}.} A queue can be simulated by stacks so that each operation takes $O(1)$ time in the worst case~\cite{Hood1981}.\footnote{\url{http://stackoverflow.com/a/5573398/1260984}.} A deque can be simulated by stacks so that each operation takes $O(1)$ time in the worst case~\cite{Chuang1993}. This problem appears on \href{https://leetcode.com/problems/implement-queue-using-stacks/}{LeetCode}.
\item implement a deque with max operation: an implementation of deque with max operation for which the worst-case time per operation is $O(1)$ exists~\cite{Gajewska1986}. The special case of implementing a queue with max operation appears on \href{https://leetcode.com/problems/sliding-window-maximum/}{LeetCode}.
\item \href{https://en.wikipedia.org/wiki/Reverse_Polish_notation#Postfix_algorithm}{evaluate reverse Polish notation}: this problem appears on \href{https://leetcode.com/problems/evaluate-reverse-polish-notation/}{LeetCode}.
\item evaluate infix expression: this problem can be solved by using the \href{https://en.wikipedia.org/wiki/Shunting-yard_algorithm}{shunting-yard algorithm}. This problem appears on \href{http://www.lintcode.com/en/problem/expression-evaluation/}{LintCode}.
\end{enumerate}

\section{Heap}
\begin{enumerate}
\item \href{https://en.wikipedia.org/wiki/Median_filter}{median filter}: one special case of finding the running medians appears on \href{https://leetcode.com/problems/find-median-from-data-stream/}{LeetCode}.
\end{enumerate}

\section{Tree}
\begin{enumerate}
\item \href{https://en.wikipedia.org/wiki/Tree_traversal#Morris_in-order_traversal_using_threading}{Morris traversal}: traverse a binary tree without using a stack. One application is to invert a binary tree, which appears on \href{https://leetcode.com/problems/invert-binary-tree/}{LeetCode}.
\item \href{https://en.wikipedia.org/wiki/Binary_tree#Encodings}{encode binary tree}: this problem appears on \href{https://leetcode.com/problems/serialize-and-deserialize-binary-tree/}{LeetCode}.
\item Rebalance binary search tree: rebalancing a binary search tree can bo done in $O(n)$ time using $O(1)$ space by using \href{https://en.wikipedia.org/wiki/Day%E2%80%93Stout%E2%80%93Warren_algorithm}{Day-Stout-Warren algorithm}. The problem of converting a sorted list to a binary search tree appears on \href{https://leetcode.com/problems/convert-sorted-list-to-binary-search-tree/}{LeetCode}.
\end{enumerate}

\section{Graph}
\begin{enumerate}
\item \href{https://en.wikipedia.org/wiki/Eulerian_path}{Euler path}: the problem of finding the smallest Euler path in the lexical order appears on \href{https://leetcode.com/problems/reconstruct-itinerary/}{LeetCode}.
\item \href{https://en.wikipedia.org/wiki/Connected-component_labeling}{connected-component labeling}: this problem appears on \href{https://leetcode.com/problems/number-of-islands/}{LeetCode}.
\item \href{https://en.wikipedia.org/wiki/Topological_sorting}{topological sort}: this problem appears on \href{http://www.lintcode.com/en/problem/topological-sorting/}{LintCode}.
\end{enumerate}

\printbibliography[heading=subbibliography]
